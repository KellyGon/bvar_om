\documentclass{beamer} % Класс презентации
% \documentclass[t]{beamer}  % выровнять текст на слайдах по верхнему краю
%  \documentclass[handout]{beamer} % Раздаточный материал
%\documentclass[aspectratio=169]{beamer} % Соотношение сторон

%\usetheme{Berkeley} % Тема оформления
%\usetheme{Bergen}
%\usetheme{Szeged}
\usetheme{Warsaw}
%\usetheme[numbers]{Statmod}

\usecolortheme{seahorse} %Цветовая схема
%\useinnertheme{circles}
%\useinnertheme{rectangles}

\usepackage{mathtext}          % русские буквы в формулах
\usepackage[T2A]{fontenc}            % внутренняя кодировка  TeX
\usepackage[utf8]{inputenc}         % кодировка исходного текста
%\usepackage{cmap}          % русский поиск в pdf
%\usepackage[english]% локализация и переносы
\usepackage{amsmath} % Математические окружения AMS
\usepackage{amsfonts} % Шрифты AMS
 \usepackage{amssymb} % Символы AMS
  \usepackage{mathtext} % Русские буквы в фомулах
  \usepackage{graphicx} % Вставить pdf- или png-файлы

  \usepackage{euscript} % Красивый шрифт
\setbeamercolor{item}{fg=blue!30}

  \usepackage{longtable}  % Длинные таблицы
  \usepackage{multirow} % Слияние строк в таблице

 \usepackage{indentfirst} % Отступ в первом абзаце.

 \newcommand*{\hm}[1]{#1\nobreak\discretionary{}%
            {\hbox{$\mathsurround=0pt #1$}}{}}
\newcommand{\argmin}{\operatornamewithlimits{argmin}}
 \usepackage{verbatim}

\usepackage{graphicx}  % Для вставки рисунков
\usepackage[update,prepend]{epstopdf} % EPS-рисунки конвертируются в PDF
\usepackage{wrapfig} % Обтекание рисунков текстом
\usepackage{booktabs}
\usepackage{changepage}


\usepackage{hyperref} % Гиперссылки



 \usepackage{etoolbox} % логические операторы

%строки ниже для вставления нумерации слайдов

\defbeamertemplate*{footline}{Warsaw} {%
\leavevmode%
\hbox{%
\begin{beamercolorbox}[wd=.5\paperwidth,ht=2.5ex,dp=1.125ex,leftskip=.3cm,rightskip=.3cm]{author in head/foot}%
\insertframenumber{}%
\hfill\insertshortauthor
\end{beamercolorbox}%
\begin{beamercolorbox}[wd=.5\paperwidth,ht=2.5ex,dp=1.125ex,leftskip=.3cm,rightskip=.3cm]{title in head/foot}%
\usebeamerfont{title in head/foot}\insertshorttitle
\end{beamercolorbox}
}%
\vskip0pt%
}
\usepackage[backend=biber, style=bwl-FU, citestyle=bwl-FU]{biblatex}
\addbibresource{bibliobase2.bib}

\DeclareMathOperator{\etr}{etr}
\DeclareMathOperator{\tr}{tr}
\DeclareMathOperator*{\argmax}{arg\,max}
%\DeclareMathOperator*{\argmin}{arg\,min}
\DeclareMathOperator{\E}{\mathbb{E}}
\DeclareMathOperator{\diag}{diag}
\DeclareMathOperator{\Var}{\mathbb{V}\mathrm{ar}}
\DeclareMathOperator{\chol}{chol}
\newcommand{\cN}{\mathcal{N}}
\newcommand{\cIW}{\mathcal{IW}}
\newcommand{\lag}{\EuScript{L}}

\newcommand{\prior}{\underline}
\newcommand{\post}{\overline}

\let\vec\relax
\DeclareMathOperator{\vec}{vec}




\author[Discussion]{Manoel Bittencourt\and Matthew Clance \and Yoseph Getachew }
\title[Trade openness and fertility rates]{Trade Openness and Fertility Rates in Africa: Panel-Data Evidence}

%\subtitle{Наша первая презентация}
%\date[Дата]{\today} % Если  \today, то можно просто не писать
%\institute[National Research University Higher School of Economics]


\date{Discussant: Oxana Malakhovskaya}

\begin{document} % Конец преамбулы, начало текста.
\begin{frame} %{1}

\titlepage

\end{frame}

\begin{frame}
\frametitle{Motivation}

\begin{itemize}
\item The paper is motivated by the Unified growth theory.
\item The aim of the paper is  to study the linkage between the trade openness and a fertility rate in Africa.
\item Implicit idea is that trade openness is a characteristic of economic development. In the regression the trade explains a part of variation that is not explained by the variation of per capita income and some other variables.
\end{itemize}
\end{frame}

\begin{frame}
\frametitle{Method and results}

Method:
\begin{itemize}
\item Panel data analysis with fixed effects, some regressions take instruments into account.
\end{itemize}

Results: 
\begin{itemize}
\item Greater trade openness is associated with lower fertility rates contrary to results of Galor and Mountford(2008)
\end{itemize}
\end{frame}


\begin{frame}
\frametitle{Suggestions}

Endogeneity

\begin{itemize}
\item it is written: 'our trade openness variable has a negative effect on fertility in Africa.' (page 3). I do not think we can talk about causation here. 
\end{itemize}


Possible negative causation

\begin{itemize}
\item it is written: 'we find that imports of manufactured goods have a significant negative effect on fertility.' (p. 3)
\end{itemize}

Another possible explanation: 
greater fertility $\to$ less per capita GDP $\to$ less import, including import of manufactured goods 

\end{frame}

\begin{frame}
\frametitle{Questions}
\begin{itemize}
\item Is it possible to take traditions (or religion or beliefs) into account? 
\item Are regressors really not collinear?  It seems that mortality rate and education level may be linked with income per capita.
\end{itemize}
\end{frame}




\begin{frame}%22
\begin{center}
THANK YOU FOR YOUR ATTENTION!
\vspace{1cm}
\end{center}
\end{frame}
\end{document}

